%%%%%%%%%%%%%%%%%%%%%%%%%%%%%%%%%%%%%%%%%
% Beamer Presentation
% LaTeX Template
% Version 1.0 (10/11/12)
%
% This template has been downloaded from:
% http://www.LaTeXTemplates.com
%
% License:
% CC BY-NC-SA 3.0 (http://creativecommons.org/licenses/by-nc-sa/3.0/)
%
%%%%%%%%%%%%%%%%%%%%%%%%%%%%%%%%%%%%%%%%%

%----------------------------------------------------------------------------------------
%	PACKAGES AND THEMES
%----------------------------------------------------------------------------------------

\documentclass{beamer}

% \mode<presentation> {

% The Beamer class comes with a number of default slide themes
% which change the colors and layouts of slides. Below this is a list
% of all the themes, uncomment each in turn to see what they look like.

%\usetheme{default}
%\usetheme{AnnArbor}
%\usetheme{Antibes}
%\usetheme{Bergen}
%\usetheme{Berkeley}
%\usetheme{Berlin}
%\usetheme{Boadilla}
%\usetheme{CambridgeUS}
%\usetheme{Copenhagen}
%\usetheme{Darmstadt}
%\usetheme{Dresden}
%\usetheme{Frankfurt}
%\usetheme{Goettingen}
%\usetheme{Hannover}
%\usetheme{Ilmenau}
%\usetheme{JuanLesPins}
%\usetheme{Luebeck}
% \usetheme{Madrid}
%\usetheme{Malmoe}
%\usetheme{Marburg}
%\usetheme{Montpellier}
\usetheme{PaloAlto}
%\usetheme{Pittsburgh}
%\usetheme{Rochester}
%\usetheme{Singapore}
%\usetheme{Szeged}
%\usetheme{Warsaw}

% As well as themes, the Beamer class has a number of color themes
% for any slide theme. Uncomment each of these in turn to see how it
% changes the colors of your current slide theme.

%\usecolortheme{albatross}
% \usecolortheme{beaver}
%\usecolortheme{beetle}
%\usecolortheme{crane}
\usecolortheme{dolphin}
%\usecolortheme{dove}
%\usecolortheme{fly}
%\usecolortheme{lily}
%\usecolortheme{orchid}
%\usecolortheme{rose}
%\usecolortheme{seagull}
% \usecolortheme{seahorse}
%\usecolortheme{whale}
%\usecolortheme{wolverine}

%\setbeamertemplate{footline} % To remove the footer line in all slides uncomment this line
% \setbeamertemplate{footline}[page number] % To replace the footer line in all slides with a simple slide count uncomment this line

\addtobeamertemplate{navigation symbols}{}{%
    \usebeamerfont{footline}%
    \usebeamercolor[fg]{footline}%
    \hspace{1em}%
    \insertframenumber/\inserttotalframenumber
}

% \setbeamertemplate{navigation symbols}{\tiny\insertframenumber}
% \setbeamertemplate{navigation symbols}{} % To remove the navigation symbols from the bottom of all slides uncomment this line
% }

\usepackage{graphicx} % Allows including images
\usepackage{booktabs} % Allows the use of \toprule, \midrule and \bottomrule in tables
%\usepackage {tikz}
% \usepackage{tkz-graph}
% \GraphInit[vstyle = Shade]
% \tikzset{
%   LabelStyle/.style = { rectangle, rounded corners, draw,
%                         minimum width = 2em, fill = yellow!50,
%                         text = red, font = \bfseries },
%   VertexStyle/.append style = { inner sep=5pt,
%                                 font = \normalsize\bfseries},
%   EdgeStyle/.append style = {->, bend left} }
% \usetikzlibrary {positioning}
%\usepackage {xcolor}
% \definecolor {processblue}{cmyk}{0.96,0,0,0}

\AtBeginSection[]
  {
     \begin{frame}<beamer>
    %  \frametitle{Plan}
     \tableofcontents[currentsection]
     \end{frame}
  }

\usepackage{subcaption}
\usepackage{multimedia}
\usepackage{dcolumn}
\usepackage{multirow}
\usepackage{url}
\usepackage{amsmath}
\usepackage{amssymb}

\newcommand{\backupbegin}{
   \newcounter{framenumberappendix}
   \setcounter{framenumberappendix}{\value{framenumber}}
}
\newcommand{\backupend}{
   \addtocounter{framenumberappendix}{-\value{framenumber}}
   \addtocounter{framenumber}{\value{framenumberappendix}} 
}

%----------------------------------------------------------------------------------------
%	TITLE PAGE
%----------------------------------------------------------------------------------------

\title[Traffic sign detection on smartphone]{Real time detection of traffic signs on mobile device} % The short title appears at the bottom of every slide, the full title is only on the title page

\author{Nicolas \textsc{Six}} % Your name
\institute[Georgia Institute of Technology] % Your institution as it will appear on the bottom of every slide, may be shorthand to save space
{
Georgia Institute of Technology\\% Your institution for the title page
School of Computer Science
\medskip
}
\date{December $3^{rd}$, 2019} % Date, can be changed to a custom date

\begin{document}

\begin{frame}
\titlepage % Print the title page as the first slide
\end{frame}

\begin{frame}{Acknowledgments}
    \begin{figure}
      \begin{center}
        \begin{subfigure}[t]{.3\linewidth}
          \centering
          \includegraphics[height=1.2\linewidth]{figures/comettee/zsoltkira.jpg}
          Dr. Zsolt \textsc{Kira}
        \end{subfigure}
        \begin{subfigure}[t]{.3\linewidth}
          \centering
          \includegraphics[height=1.2\linewidth]{figures/comettee/tsai_yichang_james.jpg}
          Professor Yi-Chang James \textsc{Tsai}
        \end{subfigure}
        \begin{subfigure}[t]{.3\linewidth}
          \centering
          \includegraphics[height=1.2\linewidth]{figures/comettee/zhahoua_wang.jpg}
          Dr. Zhaohua \textsc{Wang}
        \end{subfigure}
      \end{center}
    \end{figure}
\end{frame}{}

\begin{frame}
\frametitle{Overview} % Table of contents slide, comment this block out to remove it
\tableofcontents % Throughout your presentation, if you choose to use \section{} and \subsection{} commands, these will automatically be printed on this slide as an overview of your presentation
\end{frame}

%----------------------------------------------------------------------------------------
%	PRESENTATION SLIDES
%----------------------------------------------------------------------------------------

%------------------------------------------------

\section{Problem statement}
\begin{frame}{Traffic sign detection}
    % what and who cares
    \begin{figure}
        \centering
        \includegraphics[width=.95\linewidth]{figures/yolo_detection_example.png}
        \caption{Example of detection, from Yolo v3, classification with MobileNet.}
    \end{figure}{}
\end{frame}{}

\begin{frame}{Current procedure}
    % collect and then process -> heavy data load and delays
    \begin{figure}
        \centering
        \includegraphics[width=0.8\linewidth]{figures/self_made/current.png}
        \caption{Current procedure for automatic inventory of traffic signs}
    \end{figure}{}
\end{frame}{}

\begin{frame}{Proposed procedure}
    % can we do the same on smartphone ?
    \begin{figure}
        \centering
        \includegraphics[width=0.4\linewidth]{figures/self_made/proposed.png}
        \caption{Target procedure for automatic inventory of traffic signs}
    \end{figure}{}
\end{frame}{}

\begin{frame}{On device processing}
    % so how do they do for autonomous driving ? -> Tesla GPU (25% of car consumption at 20km/h)
    \begin{figure}
        \centering
        \includegraphics[width=0.7\linewidth]{figures/tesla-ship.jpg}
        \caption{Tesla new chip for autonomous driving, consuming 72W and providing 144 TOPS}
    \end{figure}{}
\end{frame}{}

\section{Background}
\begin{frame}{Historical traffic sign detection}
    % put some image of old papers about traffic sign detection
    \begin{figure}
      \begin{center}
        \begin{subfigure}[t]{.48\linewidth}
          \centering
          \includegraphics[width=.9\linewidth]{figures/1993trafficsigndetection.png}
          \caption{Hybrid processing to traffic sign detection proposed by Ruud Janssen, W. Ritter in 1993 \cite{janssen1993hybrid}}
        \end{subfigure}
        \begin{subfigure}[t]{.48\linewidth}
          \centering
          \includegraphics[width=.9\linewidth]{figures/vita2.png}
          \caption{Vita-II autonomous vehicle in 1994 \cite{ulmer1994vita}}
          \label{fig:resblock}
        \end{subfigure}
      \end{center}
    \end{figure}
\end{frame}{}

\begin{frame}{Generic object detection}
    % deep learning: very good results on complex task, better than human, but need powerful computer to fast.
    \begin{figure}
        \centering
        \includegraphics[width=0.8\linewidth]{figures/YOLOv3_example.png}
        \caption{Example of detection done by Yolo v3 \cite{yolov3} after training on MS COCO \cite{mscoco}}
    \end{figure}{}
\end{frame}{}

\begin{frame}{Example: SSD}
    % quick description of how SSD works.
    \begin{figure}
        \centering
        \includegraphics[width=0.8\linewidth]{figures/ssd.jpg}
        \caption{Schematic explanation of how the SSD head work \cite{liu2016ssd}}
    \end{figure}{}
\end{frame}{}

\begin{frame}{Anchors}
    \begin{figure}
        \centering
        \includegraphics[width=0.7\linewidth]{figures/anchors.png}
        \caption{Visual description of an anchor}
    \end{figure}{}
\end{frame}{}

% \begin{frame}{Generic object detection on smartphone}
%     % mobile net SSD, frame rate...
%     \begin{figure}
%         \centering
%         \includegraphics[width=0.5\linewidth]{figures/mobilnetv3_speed.png}
%         \caption{Speed and accuracy of Mobilenet V3 for object detection }
%         \label{fig:my_label}
%     \end{figure}{}
% \end{frame}{}



\section{Data}
\begin{frame}{Real data}
    % example, diversity, quantity...
    \begin{figure}
        \centering
        \includegraphics[width=0.7\linewidth]{figures/example_clear_image.jpg}
        \caption{Example of a clear image from the dataset}
    \end{figure}{}
\end{frame}{}

\begin{frame}{Real data}
    % example, diversity, quantity...
    \begin{figure}
        \centering
        \includegraphics[width=0.9\linewidth]{figures/example_bad_image.jpg}
        \caption{Example of a smartphone collected image from the dataset}
    \end{figure}{}
\end{frame}{}

\begin{frame}{Artificial data}
    % why? why not training on classical classification tasks?
    % example, count, how is it generated
    \begin{figure}
        \centering
        \includegraphics[width=0.9\linewidth]{figures/fake_data_ex.jpg}
        \caption{Example of artificially generated data}
        \label{fig:fake_im_ex}
    \end{figure}{}
\end{frame}{}





\section{Technical approach}
\subsection{Assumptions}
% add slide in between showing what each assumption imply? or after?
\begin{frame}{Aspect ratio}
    % why we can say that
    \begin{figure}
        \centering
        \includegraphics[width=0.9\linewidth]{figures/classes_diamond_nocount.jpg}
        \caption{Example of sign randomly sampled from our dataset, one example per class.}
    \end{figure}{}
\end{frame}{}

\begin{frame}{Position and size accuracy}
    % do we really care about pixel accurate position?
    \begin{figure}
        \centering
        \includegraphics[width=0.8\linewidth]{figures/bad_box_example.png}
        \caption{Example of not perfectly accurate detection boxes}
    \end{figure}{}
\end{frame}{}

\begin{frame}{Task complexity}
    % hey, did you realize that traffic sign are made to be detected? :P
    % difficult: https://farm6.staticflickr.com/5348/9675514309_de6948e5b3_z.jpg
    % src: http://cocodataset.org/#explore?id=539481
    \begin{figure}
        \centering
        \includegraphics[width=0.8\linewidth]{figures/difficult_detection.jpg}
        \caption{Example of image with an owl, from MS COCO dataset \cite{mscoco}}
    \end{figure}{}
\end{frame}{}

\subsection{Assumption implementation}
\begin{frame}{Assumption implementation}
    % predict one value -> size
    \begin{figure}
        \centering
        \includegraphics[width=0.5\linewidth]{figures/anchors.png}
        % \caption{Visual description of an anchor}
    \end{figure}{}
    \begin{itemize}
        \item Set $t_w = t_h \onslide<3->{=0}$ \pause
        \item Set $t_x = t_y = 0 \Leftrightarrow b_x = c_x + 0.5 \text{ and } b_y = c_y + 0.5$
    \end{itemize}{}
\end{frame}{}

\subsection{Structure chosen}
\begin{frame}{Structure chosen}
    % based on Mobile net first block (inspiration from what works...)
    \begin{figure}
        \centering
        \includegraphics[width=0.9\linewidth]{figures/my_arch.pdf}
        \caption{Final architecture chosen, based on MobileNet v2 first block}
    \end{figure}{}
\end{frame}{}




\section{Results}
\begin{frame}{Using the artificial data}
    % table of scores comparing architectures
    \begin{table}[]
    \centering
    \tiny
    \begin{tabular}{|c|c|c|c|c|c|c|c|}
        \hline
        \multirow{3}{*}{Block type} & \multirow{3}{*}{Filters} & \multicolumn{4}{|c|}{mAP} & \multicolumn{2}{|c|}{\multirow{2}{*}{Improvement}} \\
        & & \multicolumn{2}{|c|}{Direct Training} & \multicolumn{2}{|c|}{Fine tuning} & \multicolumn{2}{|c|}{}\\ 
        & & @25 & @50 & @25 & @50 & @25 & @50 \\ \hline
        \multirow{3}{*}{Residual} & $(8,16)$   & $0.58$ & $0.45$ & $0.68$ & $0.55$ & $+17.2\%$ &                               $+22.2\%$ \\
                                  & $(16,24)$ & $0.00$ & $0.00$ & $0.73$ & $0.61$ & $+\infty$ & $+\infty$ \\
                                  & $(32,64)$ & $0.62$ & $0.52$ & $0.82$ & $0.67$ & $+32.3\%$ & $+28.8\%$ \\ \hline
        Inverted Residual         & $(8,16)$   & $0.49$ & $0.38$ & $0.43$ & $0.30$ & $-12.24\%$ &                               $-21.05\%$\\ \hline
    \end{tabular}
    \caption{Summary of the evolution of the mAP with and without fine turning}
    \label{tab:mapFineTuning}
\end{table}
\end{frame}{}

\begin{frame}{Latency study}
    % show map graph with comparison with Yolo and table ?
    \begin{figure}
      \begin{center}
        % \begin{subfigure}[t]{.49\linewidth}
        %   \centering
        %   \includegraphics[width=0.99\linewidth]{figures/map_at_50_latency_models.png}
        %   \caption{Evolution of the mAP@50 with the latency}
        % \end{subfigure}
        % \begin{subfigure}[t]{.49\linewidth}
        %   \centering
        %   \includegraphics[width=0.99\linewidth]{figures/map_at_50_fps_models.png}
        %   \caption{Evolution of the mAP@50 with the frame per second}
        % \end{subfigure}
        \begin{subfigure}[t]{.49\linewidth}
          \centering
          \includegraphics[width=0.99\linewidth]{figures/map_at_25_latency_models.png}
          \caption{Evolution of the mAP@25 with the latency}
        \end{subfigure}
        \begin{subfigure}[t]{.49\linewidth}
          \centering
          \includegraphics[width=0.99\linewidth]{figures/map_at_25_fps_models.png}
          \caption{Evolution of the mAP@25 with the frame per second}
        \end{subfigure}
        \caption{Plots of the ratio computation speed vs accuracy for different architectures}
        \label{fig:map_to_speed}
      \end{center}
    \end{figure}
\end{frame}{}



\section{Validation}
\begin{frame}{West Westley}
    % describe ww, what is interesting or not
    \begin{figure}
        \centering
        \includegraphics[width=0.6\linewidth]{figures/westwestleymap.png}
        \caption{Map of the West Westley test road}
        \label{fig:wwmap}
    \end{figure}{}
    \small
    \begin{table}[]
        \centering
        \begin{tabular}{|c|c|c|c|c|c|c|}
            \hline
            Frame count & Sign count & TP & FP & FN & Precision & Recall \\ \hline
            $22,198$ & $34$ & $31$ & $10$ & $3$ & $76\%$ & $91\%$ \\ \hline
        \end{tabular}
        % \caption{Quantitative results for the test on West Westley. The numbers are given based on individual signs or object detected not on frame with detection.}
        \label{tab:ww_numbers}
    \end{table}{}
\end{frame}{}

\begin{frame}{West Wesley False Positive cases}
    % list some meaning full examples
    \begin{figure}
      \begin{center}
        \begin{subfigure}[t]{.49\linewidth}
          \centering
          \includegraphics[width=0.99\linewidth]{figures/examples/westwestley/FP/FP02.jpg}
          \caption{Small yellow flag}
          \label{fig:wwFlag1FP}
        \end{subfigure}
        \begin{subfigure}[t]{.49\linewidth}
          \centering
          \includegraphics[width=0.99\linewidth]{figures/examples/westwestley/FP/FP05.jpg}
          \caption{School Sign (MUTCD: S1-1)}
          \label{fig:wwSchoolFP}
        \end{subfigure}
        \begin{subfigure}[t]{.49\linewidth}
          \centering
          \includegraphics[width=0.99\linewidth]{figures/examples/westwestley/FP/FP07.jpg}
          \caption{Chevron sign}
          \label{fig:wwChevron2FP}
        \end{subfigure}
        \begin{subfigure}[t]{.49\linewidth}
          \centering
          \includegraphics[width=0.99\linewidth]{figures/examples/westwestley/FP/FP09.png}
           \caption{Back of traffic light}
          \label{fig:wwTraffficLight2FP}
        \end{subfigure}
        % \caption{Example of false positive cases collected on West Westley.}
        \label{fig:wwFPcases}
      \end{center}
    \end{figure}
\end{frame}{}

\begin{frame}{West Wesley False Negative cases}
    % list some meaning full examples
    \begin{figure}
      \begin{center}
        \begin{subfigure}[t]{.49\linewidth}
          \centering
          \includegraphics[width=0.99\linewidth]{figures/examples/westwestley/FN/FN01.jpg}
          \caption{Whitened sign}
          \label{fig:wwWhiteFN}
        \end{subfigure}
        \begin{subfigure}[t]{.49\linewidth}
          \centering
          \includegraphics[width=0.99\linewidth]{figures/examples/westwestley/FN/FN02.jpg}
          \caption{Dirty, faded sign in the shadow}
          \label{fig:wwFaddedFN}
        \end{subfigure}
        \begin{subfigure}[t]{.49\linewidth}
          \centering
          \includegraphics[width=0.99\linewidth]{figures/examples/westwestley/FN/FN03.png}
         \caption{Small often obstructed sign}
          \label{fig:wwobsFN}
        \end{subfigure}
        % \caption{Example of false negative cases collected on West Westley.}
        \label{fig:wwFNcases}
      \end{center}
    \end{figure}
\end{frame}{}

\begin{frame}{West Wesley True Positive example}
    % interesting examples
    \begin{figure}
      \begin{center}
        \begin{subfigure}[t]{.49\linewidth}
          \centering
          \includegraphics[width=0.99\linewidth]{figures/examples/westwestley/TP/TP04.jpg}
          \caption{Obstructed sign}
          \label{fig:wwObs1TP}
        \end{subfigure}
        \begin{subfigure}[t]{.49\linewidth}
          \centering
          \includegraphics[width=0.99\linewidth]{figures/examples/westwestley/TP/TP02.jpg}
          \caption{Obstructed sign}
          \label{fig:wwObs2TP}
        \end{subfigure}
        \begin{subfigure}[t]{.49\linewidth}
          \centering
          \includegraphics[width=0.99\linewidth]{figures/examples/westwestley/TP/TP05.png}
          \caption{Obstructed sign}
          \label{fig:wwObs3TP}
        \end{subfigure}
        % \caption{Example of true positive cases collected on West Westley.}
        \label{fig:wwTPcases}
      \end{center}
    \end{figure}
\end{frame}{}




\section{Conclusion}
\begin{frame}{Wrap up}
    Assumptions:
    \begin{itemize}
        \item Low position accuracy $\checkmark$
        \item Small backbone $\checkmark$
    \end{itemize}{}
\end{frame}{}

\begin{frame}{Future work}
    \begin{itemize}
        \item Implement and deploy android / iOS app \pause
        \item Experiment with higher position accuracy \pause
        \item Try full model quantification \pause
        \item Try on other signs
        % \item Explain why inverted residual bottleneck is not working
    \end{itemize}{}
\end{frame}{}

\begin{frame}{Contributions}
    % give / show the numbers
    % data generation
    \begin{figure}
      \begin{center}
        % \begin{subfigure}[t]{.49\linewidth}
        %   \centering
        %   \includegraphics[width=0.99\linewidth]{figures/map_at_50_latency_models.png}
        %   \caption{Evolution of the mAP@50 with the latency}
        % \end{subfigure}
        % \begin{subfigure}[t]{.49\linewidth}
        %   \centering
        %   \includegraphics[width=0.99\linewidth]{figures/map_at_50_fps_models.png}
        %   \caption{Evolution of the mAP@50 with the frame per second}
        % \end{subfigure}
        \begin{subfigure}[t]{.45\linewidth}
          \centering
          \includegraphics[width=0.7\linewidth]{figures/map_at_25_latency_models.png}
        %   \caption{Evolution of the mAP@25 with the latency}
        \end{subfigure}
        \begin{subfigure}[t]{.45\linewidth}
          \centering
          \includegraphics[width=0.7\linewidth]{figures/map_at_25_fps_models.png}
        %   \caption{Evolution of the mAP@25 with the frame per second}
        \end{subfigure}
        % \caption{Plots of the ratio computation speed vs accuracy for different architectures}
        \label{fig:map_to_speed}
      \end{center}
    \end{figure}
    \begin{itemize}
        \item Accurate detection of traffic sign on mobile device at above real time speed
        \item New artificial data generation method for pretraining
    \end{itemize}{}
\end{frame}{}

\begin{frame}{}
    \centering
    \Large{Questions?}
\end{frame}{}




\appendix
\backupbegin





% appendix ?
\begin{frame}{References}
    \tiny
    \bibliographystyle{plain}
    \bibliography{references} 
\end{frame}{}









\begin{frame}{mAP vs speed numbers}
    \tiny
    \begin{table}[]
        \centering
        \begin{tabular}{|c|c|c|c|c|c|r|}
            \hline
            Architecture & Input size & Filters & mAP@50 & mAP@25 & Latency (s) & FPS \\ \hline
            \multirow{ 4 }{*}{ Yolov3-tiny } & 704x416 & - & 0.92 & 0.97 & 1.738 & 0.6 \\
        	  & 576x320 & - & 0.96 & 0.98 & 1.075 & 0.9 \\
        	  & 160x288 & - & 0.77 & 0.93 & 0.252 & 4.0 \\
        	  & 96x160  & - & 0.39 & 0.70 & 0.078 & 12.8 \\
        	\hline
        	\multirow{ 4 }{*}{ \shortstack{Residual\\convolution} } & 110x200 & (6,10) & 0.48 & 0.59 & 0.009 & 113.7 \\
        	  & 110x200 & (8,16) & 0.55 & 0.68 & 0.011 & 88.7 \\
        	  & 110x200 & (16,24) & 0.61 & 0.73 & 0.020 & 48.9 \\
        	  & 110x200 & (32,64) & 0.67 & 0.82 & 0.068 & 14.8 \\
        	\hline
        	Inverted & & & & & & \\
        	residual & 110x200 & (8,16) & 0.38 & 0.49 & 0.025 & 39.4 \\
        	bottleneck & & & & & & \\
        	\hline
        \end{tabular}
        \caption{Value of the mAP, latency and frame per second for different configuration and different models}
        \label{tab:map_to_speed}
    \end{table}{}
\end{frame}{}

\begin{frame}{A feature extractor}
    % example of classification
    \begin{figure}
        \centering
        \includegraphics[width=\linewidth]{figures/vgg16.pdf}
        \caption{Example of the VGG16 architecture, on a classification task}
    \end{figure}{}
\end{frame}{}

\begin{frame}{Doing it manually}
    % how would I do if I was a CNN?
    % display filters for edges, propagation...
    % example images of detection?
  \begin{columns}
  \begin{column}{0.15\textwidth}
    \begin{figure}
        \centering
        \includegraphics[width=\linewidth]{figures/manual_filters/base_image.jpg}
        \caption{Original Image}
    \end{figure}{}
  \end{column}
  \begin{column}{0.85\textwidth}
  \begin{figure}
      \centering
      \begin{subfigure}[b]{0.22\textwidth}
          \includegraphics[width=\textwidth]{figures/manual_filters/edge_1.png}
      \end{subfigure}
      \begin{subfigure}[b]{0.22\textwidth}
          \includegraphics[width=\textwidth]{figures/manual_filters/edge_3.png}
      \end{subfigure}
      \begin{subfigure}[b]{0.22\textwidth}
          \includegraphics[width=\textwidth]{figures/manual_filters/edge_5.png}
      \end{subfigure}
      \begin{subfigure}[b]{0.22\textwidth}
          \includegraphics[width=\textwidth]{figures/manual_filters/edge_7.png}
      \end{subfigure}
      
      \begin{subfigure}[b]{0.22\textwidth}
          \includegraphics[width=\textwidth]{figures/manual_filters/edge_res_1.png}
      \end{subfigure}
      \begin{subfigure}[b]{0.22\textwidth}
          \includegraphics[width=\textwidth]{figures/manual_filters/edge_res_3.png}
      \end{subfigure}
      \begin{subfigure}[b]{0.22\textwidth}
          \includegraphics[width=\textwidth]{figures/manual_filters/edge_res_5.png}
      \end{subfigure}
      \begin{subfigure}[b]{0.22\textwidth}
          \includegraphics[width=\textwidth]{figures/manual_filters/edge_res_7.png}
      \end{subfigure}
      \caption{Edge detector convolution filters and results.}
    \end{figure}
  \end{column}
  \end{columns}
\end{frame}{}

\begin{frame}{Doing it manually}
    \begin{figure}
        \centering
        \includegraphics[width=0.8\linewidth]{figures/manual_filters/ex_im.jpg}
        \caption{Example of detection of a manually made model after training}
    \end{figure}{}
\end{frame}{}


\begin{frame}{Different blocks}
    % conv / res / residual inverted bottleneck, comparison perf/accuracy
    \begin{figure}
  \begin{center}
    \begin{subfigure}[t]{.24\linewidth}
      \centering
      \includegraphics[width=.8\linewidth]{figures/mobilenetv2_conv.png}
      \caption{2D Convolution block}
      \label{fig:convblock}
    \end{subfigure}
    \begin{subfigure}[t]{.24\linewidth}
      \centering
      \includegraphics[width=.8\linewidth]{figures/mobilenetv2_resnet.png}
      \caption{Residual convolution block}
      \label{fig:resblock}
    \end{subfigure}
    \begin{subfigure}[t]{.5\linewidth}
      \centering
      \includegraphics[width=.99\linewidth]{figures/mobilenetv2_inverted_residual_botleneck.png}
      \caption{Inverted residual block \cite{sandler2018mobilenetv2}}
      \label{fig:invertedbottleneckblock}
    \end{subfigure}
    \caption{Graphical representation of the different block used in this study, Illustrations from MobileNet V2 paper \cite{sandler2018mobilenetv2}}
  \end{center}
\end{figure}
\end{frame}{}

\begin{frame}{About input size}
    % example of input size image
    \begin{figure}
  \begin{center}
    \begin{subfigure}[t]{.4\linewidth}
      \centering
      \includegraphics[width=0.99\linewidth]{figures/im_example_full.jpg}
      \caption{$1080\times1920$}
    \end{subfigure}
    \begin{subfigure}[t]{.4\linewidth}
      \centering
      \includegraphics[width=0.99\linewidth]{figures/im_example_440x800.jpg}
      \caption{$440\times800$}
    \end{subfigure}
    \begin{subfigure}[t]{.4\linewidth}
      \centering
      \includegraphics[width=0.99\linewidth]{figures/im_example_220x400.jpg}
      \caption{$220\times400$}
    \end{subfigure}
    \begin{subfigure}[t]{.4\linewidth}
      \centering
      \includegraphics[width=0.99\linewidth]{figures/im_example_110x200.jpg}
      \caption{$110\times200$}
    \end{subfigure}
    \caption{Example of an image resized at different size and displayed at same size}
    \label{resizeExample}
  \end{center}
\end{figure}
\end{frame}{}

\begin{frame}{Input size and speed}
    % graph of input size vs speed
    \begin{figure}
  \begin{center}
    \begin{subfigure}[t]{.49\linewidth}
      \centering
      \includegraphics[width=0.99\linewidth]{figures/speed_by_nn_size_and.png}
      \caption{Latency evolution with input size}
      \label{fig:latencysize}
    \end{subfigure}
    \begin{subfigure}[t]{.49\linewidth}
      \centering
      \includegraphics[width=0.99\linewidth]{figures/speed_by_nn_size_fps_and.png}
      \caption{FPS evolution with input size}
      \label{fig:fpssize}
    \end{subfigure}
    \caption{Evolution of the prediction speed for different block type, using Tensorflow Lite version of the model on Samsung S6}
    \label{resizeExample}
  \end{center}
\end{figure}
\end{frame}{}


\begin{frame}{Neural network structure}
    \begin{table}[]
        \tiny
        \centering
        \begin{tabular}{|c|c|c|c|c|c|c|c|c|}
        \hline
        Name & Filter & Kernel & Stride & Activation & BN & Res & Input size & Input \\ \hline
        Conv1 & $8$ & $(3, 3)$ & $(2, 2)$ & ReLu6 & \checkmark &  & $(111, 201, 3)$ &  \\
        block\_0 & $8$ & $(3, 3)$ & $(1, 1)$ & ReLu6 & \checkmark & \checkmark & $(55, 100, 8)$ & Conv1 \\
        block\_1 & $16$ & $(3, 3)$ & $(2, 2)$ & ReLu6 & \checkmark &  & $(55, 100, 8)$ & block\_0 \\
        block\_2 & $16$ & $(3, 3)$ & $(1, 1)$ & ReLu6 & \checkmark & \checkmark & $(28, 50, 16)$ & block\_1 \\
        block\_3 & $16$ & $(3, 3)$ & $(1, 1)$ & ReLu6 & \checkmark & \checkmark & $(28, 50, 16)$ & block\_2 \\
        block\_4 & $16$ & $(3, 3)$ & $(2, 2)$ & ReLu6 & \checkmark &  & $(28, 50, 16)$ & block\_3 \\
        block\_5 & $16$ & $(3, 3)$ & $(1, 1)$ & ReLu6 & \checkmark & \checkmark & $(14, 25, 16)$ & block\_4 \\
        block\_6 & $16$ & $(3, 3)$ & $(1, 1)$ & ReLu6 & \checkmark & \checkmark & $(14, 25, 16)$ & block\_5 \\
        output\_1 & $2$ & $(3, 3)$ & $(1, 1)$ & Linear &  &  & $(28, 50, 16)$ & block\_3 \\
        output\_2 & $2$ & $(3, 3)$ & $(1, 1)$ & Linear &  &  & $(28, 50, 16)$ & block\_3 \\
        output\_3 & $2$ & $(3, 3)$ & $(1, 1)$ & Linear &  &  & $(14, 25, 16)$ & block\_6 \\
        output\_4 & $2$ & $(3, 3)$ & $(1, 1)$ & Linear &  &  & $(14, 25, 16)$ & block\_6 \\
        output\_5 & $2$ & $(3, 3)$ & $(1, 1)$ & Softmax &  &  & $(14, 25, 16)$ & block\_6 \\
        \hline
        \end{tabular}
        \caption{Description of the neural network structure, block by block, giving the link between the blocks and some information such as the use Batch Normalization (BN) or residual connections (Res).}
        \label{tab:nnstruct}
    \end{table}{}
\end{frame}{}

\begin{frame}{Overfitting study}
    % explain the idea
    % show graphs
    % show problems with the inverted bottleneck... + conv == res > res + not enough images
    \begin{figure}
  \begin{center}
    \begin{subfigure}[t]{.49\linewidth}
      \centering
      \includegraphics[width=0.99\linewidth]{figures/all_epoch_loss_block_type.png}
      \caption{Evolution of the training loss with epoch, for different block type}
      \label{fig:overfitloss}
    \end{subfigure}
    \begin{subfigure}[t]{.49\linewidth}
      \centering
      \includegraphics[width=0.99\linewidth]{figures/all_epoch_val_loss_block_type.png}
      \caption{Evolution of the validation loss with epoch, for different block type}
      \label{fig:overfitvalloss}
    \end{subfigure}
    \caption{Evolution of the validation and training loss for different block type}
  \end{center}
\end{figure}
\end{frame}{}

\begin{frame}{Overfitting study}
    % explain the idea
    % show graphs
    % show problems with the inverted bottleneck... + conv == res > res + not enough images
    \begin{figure}
  \begin{center}
    \begin{subfigure}[t]{.49\linewidth}
      \centering
      \includegraphics[width=0.99\linewidth]{figures/all_epoch_loss_block_type_zoomed_0_5.png}
      \caption{Evolution of the training loss with epoch, for different block type, zoomed on loss between 0 and 5}
      \label{fig:overfitloss_zoom}
    \end{subfigure}
    \begin{subfigure}[t]{.49\linewidth}
      \centering
      \includegraphics[width=0.99\linewidth]{figures/all_epoch_val_loss_block_type_zoomed_0_5.png}
      \caption{Evolution of the validation loss with epoch, for different block type, zoomed on loss between 0 and 5}
      \label{fig:overfitvalloss_zoom}
    \end{subfigure}
    \caption{Evolution of the validation and training loss for different block type}
  \end{center}
\end{figure}
\end{frame}{}



\begin{frame}{Real data}
    % 8000 images 4000 from smartphone, not much different weather, different luminosity
    \begin{figure}
        \centering
        \includegraphics[width=\linewidth]{figures/data_repartition.png}
        \caption{Count of example for each classes in the dataset, cropped to 250 for readability.}
        \label{fig:data_count}
    \end{figure}{}
\end{frame}{}

\begin{frame}{US Diamond Warning signs}
    % example of cropped signs? (useful for next slide)
    \begin{figure}
      \begin{center}
        \begin{subfigure}[t]{.3\linewidth}
          \centering
          \includegraphics[height=0.8\linewidth]{figures/W1-1_R.jpg}
          \caption{W1-1}
        \end{subfigure}
        \begin{subfigure}[t]{.3\linewidth}
          \centering
          \includegraphics[height=0.8\linewidth]{figures/W1-3_L.jpg}
          \caption{W1-3}
        \end{subfigure}
        \begin{subfigure}[t]{.3\linewidth}
          \centering
          \includegraphics[height=0.8\linewidth]{figures/W1-4_L.jpg}
          \caption{W1-4}
        \end{subfigure}
        \caption{Example of US diamond warning signs, with there MUTCD sign class.}
        \label{signExample}
      \end{center}
    \end{figure}
\end{frame}{}

\begin{frame}{Architecture search}
    % what is architecture search how it works and why we didn't do that (in some sens our architecture search was doing it manually?...)
    % tell we chose a grid search approach
    \begin{figure}
      \begin{center}
        \begin{subfigure}[t]{.45\linewidth}
          \centering
          \includegraphics[width=0.99\linewidth]{figures/darts_progress_recurrent-9.png}
        \end{subfigure}
        \begin{subfigure}[t]{.45\linewidth}
          \centering
          \includegraphics[width=0.99\linewidth]{figures/darts_progress_recurrent-19.png}
        \end{subfigure}
        \begin{subfigure}[t]{.45\linewidth}
          \centering
          \includegraphics[width=0.99\linewidth]{figures/darts_progress_recurrent-29.png}
        \end{subfigure}
        \begin{subfigure}[t]{.45\linewidth}
          \centering
          \includegraphics[width=0.99\linewidth]{figures/darts_progress_recurrent-39.png}
        \end{subfigure}
        \caption{Demo of the architecture search process as used in Darts \cite{liu2018darts}}
      \end{center}
    \end{figure}
\end{frame}{}

\begin{frame}{Testing}
    % disclaimer about validation
    % what model are we going to test
    \begin{figure}
        \centering
        \includegraphics[width=0.6\linewidth]{figures/my_arch.pdf}
        \caption{Architecture used for testing}
    \end{figure}{}
    \begin{table}[]
        \centering
        \begin{tabular}{|c|c|c|c|c|}
            \hline
            Filters & Latency (s) & FPS & mAP@25 & mAP@50 \\ \hline
            $(8,16)$ & 0.011 & 88.7 & $0.68$ & $0.55$ \\ \hline
        \end{tabular}
    \end{table}{}
\end{frame}{}

\begin{frame}{State Road 2}
    % describe sr2 what is interesting or not
    \begin{figure}
        \centering
        \includegraphics[width=0.8\linewidth]{figures/map_sr2.png}
        \caption{Map representing the test section we used on State Road 2}
        \label{fig:map_sr2}
    \end{figure}{}
\end{frame}{}

\begin{frame}{SR2 False Positive cases}
    % list some meaning full examples
    \begin{figure}
      \begin{center}
        \begin{subfigure}[t]{.49\linewidth}
          \centering
          \includegraphics[width=0.99\linewidth]{figures/examples/sr2/FP/FP_13.png}
          \caption{Chevron Sign}
          \label{fig:chevronFP}
        \end{subfigure}
        \begin{subfigure}[t]{.49\linewidth}
          \centering
          \includegraphics[width=0.99\linewidth]{figures/examples/sr2/FP/FP_09.png}
          \caption{Guard rail signalization}
          \label{fig:guardRailFP}
        \end{subfigure}
        \begin{subfigure}[t]{.49\linewidth}
          \centering
          \includegraphics[width=0.99\linewidth]{figures/examples/sr2/FP/FP_07.png}
          \caption{Work zone diamond sign}
          \label{fig:workzoneFP}
        \end{subfigure}
        \begin{subfigure}[t]{.49\linewidth}
          \centering
          \includegraphics[width=0.99\linewidth]{figures/examples/sr2/FP/FP_17.png}
          \caption{Advertisement sign}
          \label{fig:advertisementFP}
        \end{subfigure}
        % \caption{Example of False Positive case collected on State Road 2.}
        \label{fig:FPcases}
      \end{center}
    \end{figure}
\end{frame}{}

\begin{frame}{SR2 False Negative cases}
    % list some meaning full examples, 9 cases in total
    \begin{figure}
      \begin{center}
        \begin{subfigure}[t]{.49\linewidth}
          \centering
          \includegraphics[width=0.99\linewidth]{figures/examples/sr2/FN/FN_03.png}
          \caption{Light, faded sign}
        \end{subfigure}
        \begin{subfigure}[t]{.49\linewidth}
          \centering
          \includegraphics[width=0.99\linewidth]{figures/examples/sr2/FN/FN_09.png}
          \caption{Sign on the other side of the road}
          \label{fig:othersideFN}
        \end{subfigure}
        \begin{subfigure}[t]{.49\linewidth}
          \centering
          \includegraphics[width=0.99\linewidth]{figures/examples/sr2/FN/FN_07.png}
          \caption{Light, faded sign}
        \end{subfigure}
        \begin{subfigure}[t]{.49\linewidth}
          \centering
          \includegraphics[width=0.99\linewidth]{figures/examples/sr2/FN/FN_08.png}
          \caption{Light, faded sign}
        \end{subfigure}
        % \caption{Example of False Negative cases collected on State Road 2.}
        \label{fig:FNcases}
      \end{center}
    \end{figure}
\end{frame}{}

\begin{frame}{SR2 True Positive example}
    % interesting examples
    \begin{figure}
      \begin{center}
        \begin{subfigure}[t]{.49\linewidth}
          \centering
          \includegraphics[width=0.99\linewidth]{figures/examples/sr2/TP/TP_01.png}
          \caption{Temporary sign}
          \label{fig:temporaryTP}
        \end{subfigure}
        \begin{subfigure}[t]{.49\linewidth}
          \centering
          \includegraphics[width=0.99\linewidth]{figures/examples/sr2/TP/TP_09.png}
          \caption{Small sign}
          \label{fig:farobstTP}
        \end{subfigure}
        \begin{subfigure}[t]{.49\linewidth}
          \centering
          \includegraphics[width=0.99\linewidth]{figures/examples/sr2/TP/TP_05.png}
          \caption{Multiple signs}
          \label{fig:multipleTP}
        \end{subfigure}
        \begin{subfigure}[t]{.49\linewidth}
          \centering
          \includegraphics[width=0.99\linewidth]{figures/examples/sr2/TP/TP_02.png}
          \caption{Temporary sign}
          \label{fig:bigtempTP}
        \end{subfigure}
        % \caption{Example of True positive cases collected on State Road 2.}
        \label{fig:TPcases}
      \end{center}
    \end{figure}
\end{frame}{}


\backupend

\end{document}

